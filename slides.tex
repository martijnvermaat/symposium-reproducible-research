\documentclass[slidestop]{beamer}

\author{Martijn Vermaat}
% Note that the beamer theme doesn't support such long titles, so the title is
% copied verbatim in beamerthemelumc.sty. Sorry about that.
\title{Transparency and reproducibility in computational analyses}
\providecommand{\mySubTitle}{Symposium on reproducible research}
\providecommand{\myConference}{Research ICT - Solid Basis for Research}
\providecommand{\myDate}{07-07-2016}
\providecommand{\myLongDate}{LUMC, 7 July 2016}
\providecommand{\myGroup}{Human Genetics}
\providecommand{\myCenter}{Leiden University Medical Center}

\usetheme{lumc}

\begin{document}

% This disables the \pause command, handy in the editing phase.
%\renewcommand{\pause}{}

\makeTitleSlide\

\section{Introduction}

\subsection{Reproducibility}
\begin{pframe}
  Ability to duplicate an entire study.
  \begin{itemize}
    \item By the same researcher
    \item By someone else independently
  \end{itemize}

  \bigskip
  \pause
  Ability to create new work based on it.

  \bigskip
  \pause
  One of the main principles of the {\bf scientific method}.

  % reproducible research: product of academic research is paper along with
  % complete computational environment
\end{pframe}

\subsection{Computational analysis}
\begin{pframe}
  Typical ingredients:
  \begin{itemize}
    \item Data
    \item Code
    \item Results
  \end{itemize}

  \bigskip
  \pause
  These need {\bf provenance}.

  \bigskip
  \pause
  True reproducibility may also need: hardware, configuration, documentation,
  \ldots
\end{pframe}

\subsection{What we want to prevent}
\begin{pframe}
  \vskip-1.5em
  \begin{figure}[ht!]
    \centering
    \includegraphics[width=\textwidth,height=0.9\textheight,keepaspectratio]{data}
  \end{figure}
  % this is still pretty good
  % - on c:\ drive in easy to guess location
  % - seems to just have started
  % - no usb drives
  % - junk is separate
  % - notes and data are together
  % - no collaboration mess yet
  % - where are all the emails/chat logs/meeting notes
\end{pframe}

\section{Version control}

\subsection{What is it?}
\begin{pframe}
  \emph{The management of changes to documents, computer programs, large web
  sites, and other collections of information.} --- Wikipedia.
  \bigskip
  \pause

  Main features of modern version control systems:
  \begin{enumerate}
    \item Keep track of files and their changes over time
    \item Enable collaboration
  \end{enumerate}
\end{pframe}

\subsection{Main features}
\begin{pframe}
  For a single user:
  \begin{itemize}
    \item Revert files to a previous state
    \item Revert the entire project back to a previous state
    \item Review changes made over time
    \item Backup
  \end{itemize}
  \bigskip
  \pause

  For multiple users:
  \begin{itemize}
    \item A reliable way to share files between people/computers
    \item Allow multiple people working on the same project at the same time
    \item Conflict resolution
    \item See who made which changes at which time
  \end{itemize}
\end{pframe}

\subsection{Software engineering}
\begin{pframe}
  The use of version control is common practice here.

  \bigskip
  Today all software is being developed with version control.

  \bigskip
  \pause
  What is being controlled? {\bf source code}

  \bigskip
  Most version control systems are tailored to or originate from software
  engineering.
\end{pframe}

\subsection{Computational analysis}
\begin{pframe}
  Version control systems can help with provenance.

  \bigskip
  \pause
  \begin{itemize}
    \item Data
    \item Code
    \item Results
  \end{itemize}

  We can version control all of them.
\end{pframe}

\section{GitLab}

\subsection{https://git.lumc.nl}
\begin{pframe}
  User-friendly collaboration platform for the {\bf git} version control
  system.

  \bigskip
  But powerful for the technical user.

  \bigskip
  Currently serving {\bf 388 users} and {\bf 1,010 projects}.
\end{pframe}

\subsection{Collaboration}
\begin{pframe}
  \bigskip
  Fine-grained access model.
  \begin{itemize}
    \item Connected to LUMC Active Directory (your email account)
    \item Open to outside collaborators
    \item Groups and permissions per project
  \end{itemize}

  \bigskip
  \pause
  Collaborate on projects.
  \begin{itemize}
    \item Work together on computational analyses
    \item Share analysis results
    \item Issue tracking
    \item Wiki
  \end{itemize}
\end{pframe}

\subsection{Example of a project history}
\begin{pframe}
  \begin{figure}[ht!]
    \centering
    \includegraphics[width=\textwidth,height=0.7\textheight,keepaspectratio]{gitlab}
  \end{figure}
\end{pframe}

\subsection{Complete computational analyses}
\begin{pframe}
  GitLab can track code as wel as documentation.

  \bigskip
  Large data can also be tracked (but requires more technical knowledge).

  \bigskip
  Integration with IPython/Jupyter Notebook, an interactive computational
  environment.
\end{pframe}

\subsection{Example of an IPython Notebook}
\begin{pframe}
  \begin{figure}[ht!]
    \centering
    \includegraphics[width=\textwidth,height=0.8\textheight,keepaspectratio]{ipynb}
  \end{figure}
\end{pframe}

\section{Infrastructure}
\subsection{Beyond GitLab}
\begin{pframe}
  We need shared infrastructure that is secure but transparent to foster
  collaboration.

  \bigskip
  Examples:
  \begin{itemize}
    \item LUMC Shark compute cluster and storage
    \item SURFsara HPC cloud as used by RP3/BIOS consortium
    \item Reproducible research in BBMRI2
  \end{itemize}
\end{pframe}

\subsection{Tools don't guarantee success}
\begin{pframe}
  We try to supplement technical infrastructure with:
  \begin{itemize}
    \item Experience
    \item Developing best practices
    \item Education
  \end{itemize}
\end{pframe}

\section{Course}

\subsection{Code and data management with Git}
\begin{pframe}
  Introduction to working with the Git version control system.

  \bigskip
  For computational analysis.

  \bigskip
  Aimed primarily at researchers in the LUMC.

  \bigskip
  {\bf https://git.lumc.nl/humgen/gitcourse}
\end{pframe}

\makeAcknowledgementsSlide{
  \begin{tabular}{l}
    Michel Villerius\\
    Wibowo Arindrarto\\
    Szymon Kie\l{}basa\\
    Jeroen Laros\\
    Marco Roos\\
    Peter-Bram 't Hoen\\[2em]

    {\small Slides 4, 5, 10, 12 are adapted from Jeroen Laros}
  \end{tabular}
  \bigskip
}

\end{document}
